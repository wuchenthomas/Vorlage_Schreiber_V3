\documentclass[de]{RFFE_LB} 
%--------------------------- needed variables-------------------------------
\newcommand{\VAR}[1]{}
\newcommand{\AngebotsNr}{\VAR{Ang.Nr}}
\newcommand{\AngDatum}{\VAR{Ang.Dat}}
\newcommand{\LBtxtloc}{\VAR{LBloc}}
%--------------------------------------------------------------------------
\setkomavar{AngebotsNr}{\AngebotsNr}
\setkomavar{AngDatum}{\AngDatum}
\begin{document}
	\Large{\textbf{Leistungsbeschreibung}}\\
	
	\fontsize{11.0pt}{\baselineskip}{\selectfont
	\input{\LBtxtloc}\newline
%	Ziel des Projektes ist die Demonstration der Machbarkeit einer Transpondergestützten Leuchtfeuersteuerung für Windkraftanlagen. \\
%	
%	Bei den Transpondern handelt es sich um ADS-B Transponder. Die meisten Flugzeuge übertragen ihre Höheninformation, allerdings nicht alle. Daher soll mithilfe einer speziellen, nur flach über den Horizont, rundum empfangenden Antenne niedrig fliegende Flugzeuge detektiert werden. \\
%	
%	Das System soll so angelegt sein, dass es Flugzeuge, welche sich der Station nähern, ab einem bestimmten Abstand so lange ausgewertet werden (Tracking) bis diese den kritischen Bereich wieder verlassen. \\
%	
%	Zur Durchführung des Projektes sind folgende Arbeitsschritte geplant: \\
%	ADS-B Empfänger:
%	\begin{itemize}
%		\item Welche regulatorischen Vorgaben? Auf welche Standards kann zurück gegriffen werden?
%		\item Empfänger HF-Schaltung: 
%			\begin{enumerate}
%				\item Hardwaredesign,
%					\begin{enumerate}
%						\item Welche Komponenten sind verfügbar?
%						\item Schematic
%						\item Layout
%						\item Aufbau,Vermessung
%					\end{enumerate}
%				\item Antennen
%					\begin{enumerate}
%					\item Was ist möglich? Was gibt es?
%					\item Beschaffung bzw. Entwurf / Aufbau/ Vermessung
%					\end{enumerate}
%			\end{enumerate}
%	\end{itemize}
%	
%	\newpage
%	\blindtext[1]
}	
\end{document}