AGB gemäß ZVEI, Stand Juni 2011\\\newline
Allgemeine Lieferbedingungen für Erzeugnisse und Leistungen der Elektroindustrie („Grüne Lieferbedingungen" - GL) zur Verwendung im Geschäftsverkehr gegenüber Unternehmern*
\setlength{\columnsep}{1em}
\begin{multicols}{2}
%\twocolumn
	\begin{scriptsize}
	\textbf{I. Allgemeine Bestimmungen}\\
	\fontsize{9}{9}\selectfont{}
	1. Für die Rechtsbeziehungen zwischen Lieferer und Besteller im Zusammenhang mit den Lieferungen und/ oder Leistungen des Lieferers (im Folgenden: Lieferungen) gelten ausschließlich diese GL. Allgemeine Geschäftsbedingungen des Bestellers gelten nur insoweit, als der Lieferer ihnen ausdrücklich schriftlich zugestimmt hat. Für den Umfang der Lieferungen sind die beiderseitigen übereinstimmenden schriftlichen Erklärungen maßgebend.\\	
	2. An Kostenvoranschlägen, Zeichnungen und anderen Unterlagen (im Folgenden: Unterlagen) behält sich der Lieferer seine eigentums- und urheberrechtlichen Verwertungsrechte uneingeschränkt vor. Die Unterlagen dürfen nur nach vorheriger Zustimmung des Lieferers Dritten zugänglich gemacht werden und sind, wenn der Auftrag dem Lieferer nicht erteilt wird, diesem auf Verlangen unverzüglich zurückzugeben. Die Sätze 1 und 2 gelten entsprechend für Unterlagen des Bestellers; diese dürfen jedoch solchen Dritten zugänglich gemacht werden, denen der Lieferer zulässigerweise Lieferungen übertragen hat.\\	
	3. An Standardsoftware und Firmware hat der Besteller das nicht ausschließliche Recht zur Nutzung mit den vereinbarten Leistungsmerkmalen in unveränderter Form auf den vereinbarten Geräten. Der Besteller darf ohne ausdrückliche Vereinbarung eine Sicherungskopie der Standardsoftware erstellen.\\
	4. Teillieferungen sind zulässig, soweit sie dem Besteller zumutbar sind.\\
	5. Der Begriff „Schadensersatzansprüche" in diesen GL umfasst auch Ansprüche auf Ersatz vergeblicher Aufwendungen.\\
	\textbf{II. Preise, Zahlungsbedingungen und Aufrechnung}\\
	1. Die Preise verstehen sich ab Werk ausschließlich Verpackung zuzüglich der jeweils geltenden gesetzlichen Umsatzsteuer.\\
	2. Hat der Lieferer die Aufstellung oder Montage übernommen und ist nicht etwas anderes vereinbart, so trägt der Besteller neben der vereinbarten Vergütung alle erforderlichen Nebenkosten wie Reise- und Transportkosten sowie Auslösungen.\\
	3. Zahlungen sind frei Zahlstelle des Lieferers zu leisten.\\
	4. Der Besteller kann nur mit solchen Forderungen aufrechnen, die unbestritten oder rechtskräftig festgestellt sind.
	
	\textbf{III. Eigentumsvorbehalt}\\
	1. Die Gegenstände der Lieferungen (Vorbehaltsware) bleiben Eigentum des Lieferers bis zur Erfüllung sämtlicher ihm gegen den Besteller aus der Geschäftsverbindung zustehenden Ansprüche. Soweit der Wert aller Sicherungsrechte, die dem Lieferer zustehen, die Höhe aller gesicherten Ansprüche um mehr als 20\% übersteigt, wird der Lieferer auf Wunsch des Bestellers einen entsprechenden Teil der Sicherungsrechte freigeben; dem Lieferer steht die Wahl bei der Freigabe zwischen verschiedenen Sicherungsrechten zu.\\
	2. Während des Bestehens des Eigentumsvorbehalts ist dem Besteller eine Verpfändung oder Sicherungsübereignung untersagt und die Weiterveräußerung nur Wiederverkäufern im gewöhnlichen Geschäftsgang und nur unter der Bedingung gestattet, dass der Wiederverkäufer von seinem Kunden Bezahlung erhält oder den Vorbehalt macht, dass das Eigentum auf den Kunden erst übergeht, wenn dieser seine Zahlungsverpflichtungen erfüllt hat.\\
	3. Veräußert der Besteller Vorbehaltsware weiter, so tritt er bereits jetzt seine künftigen Forderungen aus der Weiterveräußerung gegen seine Kunden mit allen Nebenrechten - einschließlich etwaiger Saldoforderungen - sicherungshalber an den Lieferer ab, ohne dass es weiterer besonderer Erklärungen bedarf. Wird die Vorbehaltsware zusammen mit anderen Gegenständen weiter veräußert, ohne dass für die Vorbehaltsware ein Einzelpreis vereinbart wurde, so tritt der Besteller denjenigen Teil der Gesamtpreisforderung an den Lieferer ab, der dem vom Lieferer in Rechnung gestellten Preis der Vorbehaltsware entspricht.\\
	4. a) Dem Besteller ist es gestattet, die Vorbehaltsware zu verarbeiten oder mit anderen Gegenständen zu vermischen oder zu verbinden. Die Verarbeitung erfolgt für den Lieferer. Der Besteller verwahrt die dabei entstehende neue Sache für den Lieferer mit der Sorgfalt eines ordentlichen Kaufmanns. Die neue Sache gilt als Vorbehaltsware. b) Lieferer und Besteller sind sich bereits jetzt darüber einig, dass bei Verbindung oder Vermischung mit anderen, nicht dem Lieferer gehörenden Gegenständen dem Lieferer in jedem Fall Miteigentum an der neuen Sache in Höhe des Anteils zusteht, der sich aus dem Verhältnis des Wertes der verbundenen oder vermischten Vorbehaltsware zum Wert der übrigen Ware zum Zeitpunkt der Verbindung oder Vermischung ergibt. Die neue Sache gilt insoweit als Vorbehaltsware. c) Die Regelung über die Forderungsabtretung nach Nr. 3 gilt auch für die neue Sache. Die Abtretung gilt jedoch nur bis zur Höhe des Betrages, der dem vom Lieferer in Rechnung gestellten Wert der verarbeiteten, verbundenen oder vermischten Vorbehaltsware entspricht. d) Verbindet der Besteller die Vorbehaltsware mit Grundstücken oder beweglichen Sachen, so tritt er, ohne dass es weiterer besonderer Erklärungen bedarf, auch seine Forderung, die ihm als Vergütung für die Verbindung zusteht, mit allen Nebenrechten sicherungshalber in Höhe des Verhältnisses des Wertes der verbundenen Vorbehaltsware zu den übrigen verbundenen Waren zum Zeitpunkt der Verbindung an den Lieferer ab.\\
	5. Bis auf Widerruf ist der Besteller zur Einziehung abgetretener Forderungen aus der Weiterveräußerung befugt. Bei Vorliegen eines wichtigen Grundes, insbesondere bei Zahlungsverzug, Zahlungseinstellung, Eröffnung eines Insolvenzverfahrens, Wechselprotest oder begründeten Anhaltspunkten für eine Überschuldung oder drohende Zahlungsunfähigkeit des Bestellers, ist der Lieferer berechtigt, die Einziehungsermächtigung des Bestellers zu widerrufen. Außerdem kann der Lieferer nach vorheriger Androhung unter Einhaltung einer angemessenen Frist die Sicherungsabtretung offenlegen, die abgetretenen Forderungen verwerten sowie die Offenlegung der Sicherungsabtretung durch den Besteller gegenüber dem Kunden verlangen.\\
	6. Bei Pfändungen, Beschlagnahmen oder sonstigen Verfügungen oder Eingriffen Dritter hat der Besteller den Lieferer unverzüglich zu benachrichtigen. Bei Glaubhaftmachung eines berechtigten Interesses hat der Besteller dem Lieferer unverzüglich die zur Geltendmachung seiner Rechte gegen den Kunden erforderlichen Auskünfte zu erteilen und die erforderlichen Unterlagen auszuhändigen.\\
	7. Bei Pflichtverletzungen des Bestellers, insbesondere bei Zahlungsverzug, ist der Lieferer nach erfolglosem Ablauf einer dem Besteller gesetzten angemessenen Frist zur Leistung neben der Rücknahme auch zum Rücktritt berechtigt; die gesetzlichen Bestimmungen über die Entbehrlichkeit einer Fristsetzung bleiben unberührt. Der Besteller ist zur Herausgabe verpflichtet. In der Rücknahme bzw. der Geltendmachung des Eigentumsvorbehaltes oder der Pfändung der Vorbehaltsware durch den Lieferer liegt kein Rücktritt vom Vertrag, es sei denn, der Lieferer hätte dies ausdrücklich erklärt.\\
	\textbf{IV. Fristen für Lieferungen: Verzug}\\
	1. Die Einhaltung von Fristen für Lieferungen setzt den rechtzeitigen Eingang sämtlicher vom Besteller zu liefernden Unterlagen, erforderlichen Genehmigungen und Freigaben, insbesondere von Plänen, sowie die Einhaltung der vereinbarten Zahlungsbedingungen und sonstigen Verpflichtungen durch den Besteller voraus. Werden diese Voraussetzungen nicht rechtzeitig erfüllt, so verlängern sich die Fristen angemessen; dies gilt nicht, wenn der Lieferer die Verzögerung zu vertreten hat.\\
	2. Ist die Nichteinhaltung der Fristen zurückzuführen auf a) höhere Gewalt, z. B. Mobilmachung, Krieg, Terrorakte, Aufruhr, oder ähnliche Ereignisse (z. B. Streik, Aussperrung), b) Virus- und sonstige Angriffe Dritter auf das IT-System des Lieferers, soweit diese trotz Einhaltung der bei Schutzmaßnahmen üblichen Sorgfalt erfolgten, c) Hindernisse aufgrund von deutschen, US-amerikanischen sowie sonstigen anwendbaren nationalen, EU- oder internationalen Vorschriften des Außenwirtschaftsrechts oder aufgrund sonstiger Umstände, die vom Lieferer nicht zu vertreten sind, oder\\
	d) nicht rechtzeitige oder ordnungsgemäße Belieferung des Lieferers, verlängern sich die Fristen angemessen.\\
	3. Kommt der Lieferer in Verzug, kann der Besteller - sofern er glaubhaft macht, dass ihm hieraus ein Schaden entstanden ist - eine Entschädigung für jede vollendete Woche des Verzuges von je 0,5\%, insgesamt jedoch höchstens 5\% des Preises für den Teil der Lieferungen verlangen, der wegen des Verzuges nicht zweckdienlich verwendet werden konnte.\\
	4. Sowohl Schadensersatzansprüche des Bestellers wegen Verzögerung der Lieferung als auch Schadensersatzansprüche statt der Leistung, die über die in Nr. 3 genannten Grenzen hinausgehen, sind in allen Fällen verzögerter Lieferung, auch nach Ablauf einer dem Lieferer etwa gesetzten Frist zur Lieferung, ausgeschlossen. Dies gilt nicht, soweit in Fällen des Vorsatzes, der groben Fahrlässigkeit oder wegen der Verletzung des Lebens, des Körpers oder der Gesundheit gehaftet wird. Vom Vertrag kann der Besteller im Rahmen der gesetzlichen Bestimmungen nur zurücktreten, soweit die Verzögerung der Lieferung vom Lieferer zu vertreten ist. Eine Änderung der Beweislast zum Nachteil des Bestellers ist mit den vorstehenden Regelungen nicht verbunden.\\
	5. Der Besteller ist verpflichtet, auf Verlangen des Lieferers innerhalb einer angemessenen Frist zu erklären, ob er wegen der Verzögerung der Lieferung vom Vertrag zurücktritt oder auf der Lieferung besteht.\\
	6. Werden Versand oder Zustellung auf Wunsch des Bestellers um mehr als einen Monat nach Anzeige der Versandbereitschaft verzögert, kann dem Besteller für jeden weiteren angefangenen Monat Lagergeld in Höhe von 0,5\% des Preises der Gegenstände der Lieferungen, höchstens jedoch insgesamt 5\%, berechnet werden. Der Nachweis höherer oder niedrigerer Lagerkosten bleibt den Vertragsparteien unbenommen.
	
	\textbf{V. Gefahrübergang}\\
	1. Die Gefahr geht auch bei frachtfreier Lieferung wie folgt auf den Besteller über: a) bei Lieferung ohne Aufstellung oder Montage, wenn sie zum Versand gebracht oder abgeholt worden ist. Auf Wunsch und Kosten des Bestellers wird die Lieferung vom Lieferer gegen die üblichen Transportrisiken versichert; b) bei Lieferung mit Aufstellung oder Montage am Tage der Übernahme in eigenen Betrieb oder, soweit vereinbart, nach erfolgreichem Probebetrieb.\\
	2. Wenn der Versand, die Zustellung, der Beginn, die Durchführung der Aufstellung oder Montage, die Übernahme in eigenen Betrieb oder der Probebetrieb aus vom Besteller zu vertretenden Gründen verzögert wird oder der Besteller aus sonstigen Gründen in Annahmeverzug kommt, so geht die Gefahr auf den Besteller über.
	
	\textbf{VI. Aufstellung und Montage}\\
	Für die Aufstellung und Montage gelten, soweit nichts anderes schriftlich vereinbart ist, folgende Bestimmungen:\\
	1. Der Besteller hat auf seine Kosten zu übernehmen und rechtzeitig zu stellen: a) alle Erd-, Bau- und sonstigen branchenfremden Nebenarbeiten einschließlich der dazu benötigten Fach- und Hilfskräfte, Baustoffe und Werkzeuge, b) die zur Montage und Inbetriebsetzung erforderlichen Bedarfsgegenstände und - stoffe, wie Gerüste, Hebezeuge und andere Vorrichtungen, Brennstoffe und Schmiermittel, c) Energie und Wasser an der Verwendungsstelle einschließlich der Anschlüsse, Heizung und Beleuchtung, d) bei der Montagestelle für die Aufbewahrung der Maschinenteile, Apparaturen, Materialien, Werkzeuge usw. genügend große, geeignete, trockene und verschließbare Räume und für das Montagepersonal angemessene Arbeits- und Aufenthaltsräume ein-schließlich den Umständen angemessener sanitärer Anlagen; im Übrigen hat der Besteller zum Schutz des Besitzes des Lieferers und des Montagepersonals auf der Baustelle die Maßnahmen zu treffen, die er zum Schutz des eigenen Besitzes ergreifen würde, e) Schutzkleidung und Schutzvorrichtungen, die infolge besonderer Umstände der Montagestelle erforderlich sind. Vor Beginn der Montagearbeiten hat der Besteller die nötigen Angaben über die Lage verdeckt geführter Strom-, Gas-, Wasserleitungen oder ähnlicher Anlagen sowie die erforderlichen statischen Angaben unaufgefordert zur Verfügung zu stellen.\\
	2. Vor Beginn der Aufstellung oder Montage müssen sich die für die Aufnahme der Arbeiten erforderlichen Beistellungen und Gegenstände an der Aufstellungs- oder Montagestelle befinden und alle Vorarbeiten vor Beginn des Aufbaues so weit fortgeschritten sein, dass die Aufstellung oder Montage vereinbarungsgemäß begonnen und ohne Unterbrechung durchgeführt werden kann. Anfuhrwege und der Aufstellungs- oder Montageplatz müssen geebnet und geräumt sein.\\
	3. Verzögern sich die Aufstellung, Montage oder Inbetriebnahme durch nicht vom Lieferer zu vertretende Umstände, so hat der Besteller in angemessenem Umfang die Kosten für Wartezeit und zusätzlich erforderliche Reisen des Lieferers oder des Montagepersonals zu tragen.\\
	4. Der Besteller hat dem Lieferer wöchentlich die Dauer der Arbeitszeit des Montagepersonals sowie die Beendigung der Aufstellung, Montage oder Inbetriebnahme unverzüglich zu bescheinigen.\\
	5. Verlangt der Lieferer nach Fertigstellung die Abnahme der Lieferung, so hat sie der Besteller innerhalb von zwei Wochen vorzunehmen. Der Abnahme steht es gleich, wenn der Besteller die Zweiwochenfrist verstreichen lässt oder wenn die Lieferung - gegebenenfalls nach Abschluss einer vereinbarten Testphase - in Gebrauch genommen worden ist.
	
	\textbf{VII. Entgegennahme}\\
	Der Besteller darf die Entgegennahme von Lieferungen wegen unerheblicher Mängel nicht verweigern.\\
	\textbf{VIII. Sachmängel}\\
	Für Sachmängel haftet der Lieferer wie folgt:\\
	1. Alle diejenigen Teile oder Leistungen sind nach Wahl des Lieferers unentgeltlich nachzubessern, neu zu liefern oder neu zu erbringen, die einen Sachmangel aufweisen, sofern dessen Ursache bereits im Zeitpunkt des Gefahrübergangs vorlag.\\
	2. Ansprüche auf Nacherfüllung verjähren in 12 Monaten ab gesetzlichem Verjährungsbeginn; Entsprechendes gilt für Rücktritt und Minderung. Diese Frist gilt nicht, soweit das Gesetz gemäß §§ 438 Abs. 1 Nr. 2 (Bauwerke und Sachen für Bauwerke), 479 Abs. 1 (Rückgriffsanspruch) und 634a Abs. 1 Nr. 2 (Baumängel) BGB längere Fristen vorschreibt, bei Vorsatz, arglistigem Verschweigen des Mangels sowie bei Nichteinhaltung einer Beschaffenheitsgarantie. Die gesetzlichen Regelungen über Ablaufhemmung, Hemmung und Neubeginn der Fristen bleiben unberührt.\\
	3. Mängelrügen des Bestellers haben unverzüglich schriftlich zu erfolgen.\\
	4. Bei Mängelrügen dürfen Zahlungen des Bestellers in einem Umfang zurückbehalten werden, die in einem angemessenen Verhältnis zu den aufgetretenen Sachmängeln stehen. Der Besteller kann Zahlungen nur zurückbehalten, wenn eine Mängelrüge geltend gemacht wird, über deren Berechtigung kein Zweifel bestehen kann. Ein Zurückbehaltungsrecht des Bestellers besteht nicht, wenn seine Mängelansprüche verjährt sind.
	Erfolgte die Mängelrüge zu Unrecht, ist der Lieferer berechtigt, die ihm entstandenen Aufwendungen vom Besteller ersetzt zu verlangen.\\
	5. Dem Lieferer ist Gelegenheit zur Nacherfüllung innerhalb angemessener Frist zu gewähren.\\
	6. Schlägt die Nacherfüllung fehl, kann der Besteller - unbeschadet etwaiger Schadensersatzansprüche gemäß Nr. 10 - vom Vertrag zurücktreten oder die Vergütung mindern.\\
	7. Mängelansprüche bestehen nicht bei nur unerheblicher Abweichung von der vereinbarten Beschaffenheit, bei nur unerheblicher Beeinträchtigung der Brauchbarkeit, bei natürlicher Abnutzung oder Schäden, die nach dem Gefahrübergang infolge fehlerhafter oder nachlässiger Behandlung, übermäßiger Beanspruchung, ungeeigneter Betriebsmittel, mangelhafter Bauarbeiten, ungeeigneten Baugrundes oder die aufgrund besonderer äußerer Einflüsse entstehen, die nach dem Vertrag nicht vorausgesetzt sind, sowie bei nicht reproduzierbaren Softwarefehlern. Werden vom Besteller oder von Dritten unsachgemäß Änderungen oder Instandsetzungsarbeiten vorgenommen, so bestehen für diese und die daraus entstehenden Folgen ebenfalls keine Mängelansprüche.\\
	8. Ansprüche des Bestellers wegen der zum Zweck der Nacherfüllung erforderlichen Aufwendungen, insbesondere Transport-, Wege-, Arbeits- und Materialkosten, sind ausgeschlossen, soweit die Aufwendungen sich erhöhen, weil der Gegenstand der Lieferung nachträglich an einen anderen Ort als die Niederlassung des Bestellers verbracht worden ist, es sei denn, die Verbringung entspricht seinem bestimmungsgemäßen Gebrauch.\\
	9. Rückgriffsansprüche des Bestellers gegen den Lieferer gemäß § 478 BGB (Rückgriff des Unternehmers) bestehen nur insoweit, als der Besteller mit seinem Abnehmer keine über die gesetzlichen Mängelansprüche hinausgehenden Vereinbarungen getroffen hat. Für den Umfang des Rückgriffsanspruchs des Bestellers gegen den Lieferer gemäß § 478 Abs. 2 BGB gilt ferner Nr. 8 entsprechend.\\
	10. Schadensersatzansprüche des Bestellers wegen eines Sachmangels sind ausgeschlossen. Dies gilt nicht bei arglistigem Verschweigen des Mangels, bei Nichteinhaltung einer Beschaffenheitsgarantie, bei Verletzung des Lebens, des Körpers oder der Gesundheit und bei einer vorsätzlichen oder grob fahrlässigen Pflichtverletzung des Lieferers. Eine Änderung der Beweislast zum Nachteil des Bestellers ist mit den vorstehenden Regelungen nicht verbunden. Weitergehende oder andere als in diesem Art. VIII geregelten Ansprüche des Bestellers wegen eines Sachmangels sind ausgeschlossen.\\
	\textbf{IX. Gewerbliche Schutzrechte und Urheberrechte; Rechtsmängel}\\
	1. Sofern nicht anders vereinbart, ist der Lieferer verpflichtet, die Lieferung lediglich im Land des Lieferorts frei von gewerblichen Schutzrechten und Urheberrechten Dritter (im Folgenden: Schutzrechte) zu erbringen. Sofern ein Dritter wegen der Verletzung von Schutzrechten durch vom Lieferer erbrachte, vertragsgemäß genutzte Lieferungen gegen den Besteller berechtigte Ansprüche erhebt, haftet der Lieferer gegenüber dem Besteller innerhalb der in Art. VIII Nr. 2 bestimmten Frist wie folgt: a) Der Lieferer wird nach seiner Wahl auf seine Kosten für die betreffenden Lieferungen entweder ein Nutzungsrecht erwirken, sie so ändern, dass das Schutzrecht nicht verletzt wird, oder austauschen. Ist dies dem Lieferer nicht zu angemessenen Bedingungen möglich, stehen dem Besteller die gesetzlichen Rücktritts- oder Minderungsrechte zu. b) Die Pflicht des Lieferers zur Leistung von Schadensersatz richtet sich nach Art. XII. c) Die vorstehend genannten Verpflichtungen des Lieferers bestehen nur, soweit der Besteller den Lieferer über die vom Dritten geltend gemachten Ansprüche unverzüglich schriftlich verständigt, eine Verletzung nicht anerkennt und dem Lieferer alle Abwehrmaßnahmen und Vergleichsverhandlungen vorbehalten bleiben. Stellt der Besteller die Nutzung der Lieferung aus Schadensminderungs- oder sonstigen wichtigen Gründen ein, ist er verpflichtet, den Dritten darauf hinzuweisen, dass mit der Nutzungseinstellung kein Anerkenntnis einer Schutzrechtsverletzung verbunden ist.\\
	2. Ansprüche des Bestellers sind ausgeschlossen, soweit er die Schutzrechtsverletzung zu vertreten hat.\\
	3. Ansprüche des Bestellers sind ferner ausgeschlossen, soweit die Schutzrechtsverletzung durch spezielle Vorgaben des Bestellers, durch eine vom Lieferer nicht voraussehbare Anwendung oder dadurch verursacht wird, dass die Lieferung vom Besteller verändert oder zusammen mit nicht vom Lieferer gelieferten Produkten eingesetzt wird.\\
	4. Im Falle von Schutzrechtsverletzungen gelten für die in Nr.1a) geregelten Ansprüche des Bestellers im Übrigen die Bestimmungen des Art. VIII Nr. 4, 5 und 9 entsprechend.\\
	5. Bei Vorliegen sonstiger Rechtsmängel gelten die Bestimmungen des Art. VIII entsprechend.\\
	6. Weitergehende oder andere als die in diesem Art. IX geregelten Ansprüche des Bestellers gegen den Lieferer und dessen Erfüllungsgehilfen wegen eines Rechtsmangels sind ausgeschlossen.\\
	\textbf{X. Erfüllungsvorbehalt}\\
	1. Die Vertragserfüllung steht unter dem Vorbehalt, dass keine Hindernisse aufgrund von deutschen, US-amerikanischen sowie sonstigen anwendbaren nationalen, EU- oder internationalen Vorschriften des Außenwirtschaftsrechts sowie keine Embargos oder sonstige Sanktionen entgegenstehen.\\
	2. Der Besteller ist verpflichtet, alle Informationen und Unterlagen beizubringen, die für die Ausfuhr, Verbringung bzw. Einfuhr benötigt werden.
	
	\textbf{XI. Unmöglichkeit, Vertragsanpassung}\\
	1. Soweit die Lieferung unmöglich ist, ist der Besteller berechtigt, Schadensersatz zu verlangen, es sei denn, dass der Lieferer die Unmöglichkeit nicht zu vertreten hat. Jedoch beschränkt sich der Schadensersatzanspruch des Bestellers auf 10\% des Wertes desjenigen Teils der Lieferung, der wegen der Unmöglichkeit nicht zweckdienlich verwendet werden kann. Diese Beschränkung gilt nicht, soweit in Fällen des Vorsatzes, der groben Fahrlässigkeit oder wegen der Verletzung des Lebens, des Körpers oder der Gesundheit gehaftet wird; eine Änderung der Beweislast zum Nachteil des Bestellers ist hiermit nicht verbunden. Das Recht des Bestellers zum Rücktritt vom Vertrag bleibt unberührt.\\
	2. Sofern Ereignisse im Sinne von Art. IV Nr. 2 a) bis c) die wirtschaftliche Bedeutung oder den Inhalt der Lieferung erheblich verändern oder auf den Betrieb des Lieferers erheblich einwirken, wird der Vertrag unter Beachtung von Treu und Glauben angemessen angepasst. Soweit dies wirtschaftlich nicht vertretbar ist, steht dem Lieferer das Recht zu, vom Vertrag zurückzutreten. Gleiches gilt, wenn erforderliche Ausfuhrgenehmigungen nicht erteilt werden oder nicht nutzbar sind. Will er von diesem Rücktrittsrecht Gebrauch machen, so hat er dies nach Erkenntnis der Tragweite des Ereignisses unverzüglich dem Besteller mitzuteilen und zwar auch dann, wenn zunächst mit dem Besteller eine Verlängerung der Lieferzeit vereinbart war.\\
	\textbf{XII. Sonstige Schadensersatzansprüche}\\
	1. Soweit nicht anderweitig in diesen GL geregelt, sind Schadensersatzansprüche des Bestellers, gleich aus welchem Rechtsgrund, insbesondere wegen Verletzung von Pflichten aus dem Schuldverhältnis und aus unerlaubter Handlung, ausgeschlossen.\\
	2. Dies gilt nicht, soweit wie folgt gehaftet wird: a) nach dem Produkthaftungsgesetz, b) bei Vorsatz, c) bei grober Fahrlässigkeit von Inhabern, gesetzlichen Vertretern oder leitenden Angestellten, d) bei Arglist, e) bei Nichteinhaltung einer übernommenen Garantie, f) wegen der schuldhaften Verletzung des Lebens, des Körpers oder der Gesundheit, oder g) wegen der schuldhaften Verletzung wesentlicher Vertragspflichten. Der Schadensersatzanspruch für die Verletzung wesentlicher Vertragspflichten ist jedoch auf den vertragstypischen, vorhersehbaren Schaden begrenzt, soweit nicht ein anderer der vorgenannten Fälle vorliegt.\\
	3. Eine Änderung der Beweislast zum Nachteil des Bestellers ist mit den vorstehenden Regelungen nicht verbunden.\\
	\textbf{XIII. Gerichtsstand und anwendbares Recht}\\
	1. Alleiniger Gerichtsstand ist, wenn der Besteller Kaufmann ist, bei allen aus dem Vertragsverhältnis unmittelbar oder mittelbar sich ergebenden Streitigkeiten der Sitz des Lieferers. Der Lieferer ist jedoch auch berechtigt, am Sitz des Bestellers zu klagen.\\
	2. Dieser Vertrag einschließlich seiner Auslegung unterliegt deutschem Recht unter Ausschluss des Übereinkommens der Vereinten Nationen über Verträge über den internationalen Warenkauf (CISG).\\
	\textbf{XIV. Verbindlichkeit des Vertrages}\\
	Der Vertrag bleibt auch bei rechtlicher Unwirksamkeit einzelner Bestimmungen in seinen übrigen Teilen verbindlich. Das gilt nicht, wenn das Festhalten an dem Vertrag eine unzumutbare Härte für eine Partei darstellen würde.\\\newline
	
	* Unverbindliche Konditionenempfehlung des ZVEI - Zentralverband Elektrotechnik- und Elektronikindustrie e.V. - Stand: Juni 2011\\
	
	© 2011 ZVEI • Zentralverband Elektrotechnik- und Elektronikindustrie e.V.\\
	Lyoner Straße 9\\
	60528 Frankfurt am Main\\
	All rights reserved
	
	\end{scriptsize}
\end{multicols}